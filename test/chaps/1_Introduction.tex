\chapter{绪论}
\section{研究背景}
在现实的物理世界中,涉及到的传播信号(如语音、超声波、无线电信号等)扮演着重要的角色。因此对于这些信号的处理及利用一直以来都是人类科学研究的重要领域。在相当长的一段时间内,对于这些信号的处理都是基于单个传感器,以语音信号为例,典型应用就是采用单一的麦克风实现对语音信号的拾取。随着阵列信号处理技术的出现,其显现出了相比单传感器技术的极大优势,其中几个主要方面包括:
\section{研究意义}
稀/微阵列的典型应用主要考虑了手持移动终端应用、家庭信息设备应用、传感器网络应用以及空间装备应用。下面将围绕这四个典型应用领域详细地介绍本论文研究的意义。
\section{研究现状}
本论文研究的基于稀/微阵列的信号处理,从大的技术方向来讲还是属于阵列信号处理的范畴,因此很多阵列信号处理的理论和方法在稀/微阵列中也是适用的。本论文将从子空间波达方向估计、虚拟阵列、传感器网络定位及宽带信号增强这四个阵列信号处理所涉及到的典型领域进行文献回顾,这几个方向也与本论文的研究内容相一致。同时,在这些领域所涉及到的与稀/微阵列相关的内容也将进行描述。
\subsection{子空间波达方向估计技术的研究现状}
波达方向估计即对信号的来波方向进行估计,可简称为DOA(Direction Of Arrival)估计,属于空间谱估计的范畴,本论文如没有特别说明,空间谱估计即指波达方向估计。最早的基于阵列的空间谱估计算法可以追溯到常规波束形成(BMF,BeaM Forming)算法[1],也叫做Bartlett波束形成算法。该算法是时域傅里叶谱估计在空域的一种简单扩展,即用空域各阵列数据代替传统时域处理的时域数据。与时域傅里叶变换一样,空域分辨力受到空域傅里叶限即瑞利限的限制,故不能分辨一个波束宽度内的多个信号源。接着突破瑞利限的空间谱估计方法如Capon波束形成技术得到了研究[2]。二十世纪七十年代起利用子空间分解的算法开始兴起,最具代表性的经典算法有多重信号分类算法[3]、旋转不变信号子空间算法[4]以及它们的各类变种算法。二十世纪八十年代后期又出现了一批子空间拟合类算法,比较典型的有最大似然算法[5]、子空间拟合算法[6]等。


由于基于子空间理论的空间谱估计方法可以突破瑞利限,具备超分辨能力,因此有很好的估计性能,且相比最大似然等算法计算复杂度更低。此外,基于子空间理论的空间谱估计方法还可以运用包括阵列扩展在内的多种技术拓展其应用。因此,自该类方法出现伊始,就一直是空间谱估计领域的重要研究方向。已有文献对基于子空间理论的波达方向估计方法进行回顾[7],但鉴于近些年来在该领域的一些最新进展,且子空间方法将作为本论文的重点研究内容,因此将对基于子空间理论的波达方向估计研究进行独立的文献回顾。


在本节的子空间波达方向估计研究文献回顾中,主要对多重信号分类算法、旋转不变子空间算法以及它们的各变种算法进行描述。除此之外,基于子空间理论,还涌现出了许多其他波达方向估计算法,也将一并提及。在本节的最后对子空间获取方法的相关文献进行了回顾。

\subsubsection{多重信号分类(MUSIC)类空间谱估计方法研究}
多重信号分类算法,或称为MUSIC(MUltiple SIgnal Classification)算法,是子空间类空间谱估计方法的典型代表之一,它采用对接收信号所构造的协方差矩阵进行特征分解的方法求得信号子空间和噪声子空间,并基于噪声子空间进行搜索以获得空间谱。Schmidt最早于1982年在其博士学位论文中提出该方法,并命名为MUSIC(MUltiple SIgnal Characterization)[3],于1986年又将其更名为MUltiple SIgnal Classification并发表[8]。在二十世纪八十年代末至九十年代初,对多重信号分类方法的理论研究及分析主要还是集中在传统MUSIC算法本身。进入二十一世纪,对该类方法的理论分析已是针对一些改进算法而进行,如针对MUSIC-like算法的深入分析[9]。在多重信号分类方法的应用推广研究中,降低运算复杂度是重要方面,但算法在降低运算度的同时,估计性能往往也会降低。而E. A. Santiago 等人提出用一种迭代的子空间方法进行多个源的DOA估计,在低样点数和低SNR的情况下,相比已有的root-MUSIC算法,在具有更低计算量的同时,还能保持更好的性能[10]。实数操作相比复数操作有更低的运算要求,因此变复数运算为实数运算也是降低运算复杂度的一条有效途径[11]。除了降低运算复杂度外,具备更稳健特性的改进算法[12]也有益于该类算法的应用推广。
