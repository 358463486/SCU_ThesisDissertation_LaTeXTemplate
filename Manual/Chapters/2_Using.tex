%!TEX root = ../Manual.tex
\chapter{模版的使用}
\label{Chap_using}
\section{功能}
本节主要介绍本模版所依赖的类库和提供的功能/命令。
\subsection{依赖的类库}
本模版依赖的类库/宏包如\autoref{table_RequiredPackages}所示:
\begin{table}[H]
	\centering
	\caption{本模版依赖的类库/宏包}
	\label{table_RequiredPackages}
	\zihao{5}
	\begin{tabular*}{\textwidth}{l@{\extracolsep{\fill}}p{0.7\textwidth}}
		\toprule
		\textbf{宏包名称} & \textbf{功能描述}     \\
		\midrule
		\verb|ctexbook|\cite{Packages_CTeX}    & \CTeX~书籍类,支持中文书籍类的~\LaTeX~框架。本模版是以此为核心构建的。\\
		\verb|ifthen|\cite{Packages_ifthen} & 提供增强型的逻辑判断功能。 \\
		\verb|graphicx|\cite{Packages_graphicx} & 提供图片插入及其增强功能,支持~pdf~、~eps~等格式。 \\
		\verb|anysize|\cite{Packages_anysize} & 支持自定义纸张大小,本模版的16开页面需此宏支持。 \\
		\verb|epstopdf|\cite{Packages_epstopdf} & 支持将~eps~转换为~pdf~,并能够跨目录访问。 \\
		\verb|hyperref|\cite{Packages_hyperref} & 支持文档导航标签及超链接功能。 \\
		\verb|tocloft|\cite{Packages_tocloft} & 支持自定义目录样式。 \\
		\verb|tocbibind|\cite{Packages_tocbibind} & 支持在目录中显示参考文献、附录等项目。 \\
		\verb|caption2|\cite{Packages_caption2} & 支持自定义题注样式。 \\
		\verb|natbib|\cite{Packages_natbib} & 支持自定义参考文献编号样式,提供编号排序和分类。 \\
		\verb|enumitem|\cite{Packages_enumitem} & 支持自定义列表环境。 \\
		\verb|amsmath|\cite{Packages_amsmath} & 支持~\AmS~通用的数学表达方式。 \\
		\verb|amsfonts|\cite{Packages_amsfonts} & 支持~\AmS~通用的特殊数学字体,如~$\mathbb{C}$~、~$\mathbb{R}$~等。 \\
		\verb|amsthm|\cite{Packages_amsthm} & 支持~\AmS~通用的数学定理环境。 \\
		\verb|mathtools|\cite{Packages_mathtools} & 为~\AmS~数学表达式提供扩展,如公式的多行环境。 \\
		\verb|amssymb|\cite{Packages_amssymb} & 支持~\AmS~通用的特殊符号。 \\
		\verb|float|\cite{Packages_float} & 支持图表类浮动对象的扩展设置。 \\
		\verb|booktabs|\cite{Packages_booktabs} & 支持三线表等专业表格。 \\
		\bottomrule
	\end{tabular*}
\end{table}
\subsection{提供的选项}
本模版以~\verb|ctexbook|~文档类为基础,提供~\verb|<degree>|~选项用于选择学位论文类别,其余选项~\verb|<ctexbook_opt>|~均会被传递给~\verb|ctexbook|~文档类。


加载本模版定义的文档类的命令为:
\fvset{fontsize=\zihao{-5}}
\begin{Verbatim}[gobble=1,frame=single]
	\documentclass[<degree>,<ctexbook_opt1>,...,<ctexbook_optXX>]{../Template/scuthesis}
\end{Verbatim}
其中,~\verb|<degree>|~可用选项为~\verb|doctor|~、~\verb|master|~和~\verb|bachelor|\footnote{虽然此选项代表学士学位论文选项,但并未针对其论文格式要求作出适配。}~,分别代表博士、硕士和学士学位论文。值得注意的是,一般推荐使用~\verb|UTF-8|~编码撰写论文,因此建议设置~\verb|<ctexbook_opt1>|~为~\verb|UTF-8|~。其他有关~\verb|ctexbook|~文档类的选项请参考相关文档\cite{Packages_CTeX}。


本手册加载~\emph{scuthesis}~文档类的命令为:
\begin{Verbatim}[gobble=1,frame=single]
	\documentclass[master,UTF8,hyperref]{../Template/scuthesis}
\end{Verbatim}

\subsection{提供的命令}
本模版提供的带参数命令如\autoref{table_ProvidedCommandsWithPar}所示。这种带参数的命令一般用以下方式调用:
\fvset{fontsize=\zihao{5}}
\begin{Verbatim}[gobble=1,frame=single]
	\command{<parameter1>}{<parameter2>}...{<parameterXX>}
\end{Verbatim}
其中,~\verb|<parameter>|~指输入的参数,用~\verb|{ }|~包含。


\begin{center}
	\topcaption{\emph{scuthesis}~提供的带参数命令}
	\label{table_ProvidedCommandsWithPar}
	\tablefirsthead{
		\toprule
		\multicolumn{1}{l}{\textbf{命令}} & \multicolumn{1}{l}{\textbf{功能描述}}\\
		\midrule
	}
	\tablehead{
		\multicolumn{2}{l}{\emph{续\autoref{table_ProvidedCommandsWithPar}:}}\\
		\toprule
		\multicolumn{1}{l}{\textbf{命令}} & \multicolumn{1}{l}{\textbf{功能描述}}\\
		\midrule
	}
	\tabletail{
		\bottomrule
		\multicolumn{2}{r}{\emph{见下页}}\\
	}
	\tablelasttail{\bottomrule}
	\begin{supertabular*}{\textwidth}{l@{\extracolsep{\fill}}p{0.7\textwidth}}
		\verb|\title| & 设定论文中文标题\\
		\verb|\ENGtitle| & 设定论文英文标题\\
		\verb|\author| & 设定论文作者中文姓名\\
		\verb|\ENGauthor| & 设定论文作者英文姓名\\
		\verb|\accomplishdate| & 设定论文完成日期\\
		\verb|\school| & 设定所属学院\\
		\verb|\supervisor| & 设定导师中文姓名\\
		\verb|\ENGsupervisor| & 设定导师英文姓名\\
		\verb|\major| & 设定专业中文名\\
		\verb|\ENGmajor| & 设定专业英文名\\
		\verb|\direction| & 设定研究方向中文名\\
		\verb|\ENGdirection| & 设定研究方向英文名\\
		\verb|\defensedate| & 设定答辩日期\\
		\verb|\keywords| & 设定中文关键词\\
		\verb|\ENGkeywords| & 设定英文关键词\\
		\verb|\university| & 设定大学中文名称\\
		\verb|\ENGuniversity| & 设定大学英文名称\\
		\verb|\fillinblank| & 双参数命令,用于在指定字段下加特定长度的下划线。第一个参数为下划线长度,第二个参数为输出字段\\

	\end{supertabular*}
\end{center}


如需设定论文的中英文标题,则在文章正式开始前输入:
\begin{Verbatim}[gobble=1,frame=single]
	\title{四川大学学位论文~\LaTeX~模版 Ver. 0.1}
	\ENGtitle{The SCU Dissertation \LaTeX ~ Class Ver. 0.1}
\end{Verbatim}


如需设定论文的中英文作者姓名,则在文章正式开始前输入:
\begin{Verbatim}[gobble=1,frame=single]
	\author{Legendary L.}
	\ENGauthor{Legendary L.}
\end{Verbatim}


或需得到~3cm~下划线上的\fillinblank{3cm}{四川大学},需输入:
\begin{Verbatim}[gobble=1,frame=single]
	\fillinblank{3cm}{四川大学}
\end{Verbatim}


本模版提供的不带参数命令如\autoref{table_ProvidedCommandsWithoutPar}所示:
\begin{table}[h]
	\caption{\emph{scuthesis}~提供的不带参数命令}
	\label{table_ProvidedCommandsWithoutPar}
	\begin{tabular*}{\textwidth}{l@{\extracolsep{\fill}}p{0.6\textwidth}}
		\toprule
		\textbf{命令} & \textbf{功能描述} \\
		\midrule
		\verb|\maketitle| & 根据设置字段自动生成论文封面\\
		\verb|\maketoc| & 根据章节信息自动生成目录\\
		\verb|\makechaptertitlecenter| & 使章节标题居中\\
		\verb|\makechaptertitleleft| & 使章节标题居左\\
		\verb|\autograph| & 生成如学位论文版权使用授权书样式的签名栏\\
		\bottomrule
	\end{tabular*}
\end{table}


如需生成封面,则在输入完必要信息后,使用~\verb|\maketitle|~命令生成。或若需将章节标题居左,就只需要在需要居左的章节前加~\verb|\makechaptertitleleft|~,如:
\begin{Verbatim}
	\makechaptertitleleft	% 章节标题居左
	\chapter{绪论}	% “绪论”章节
	..............
	\chapter{问题模型的建立}	% “问题模型的建立”章节
	..............
\end{Verbatim}


\subsection{提供的环境}
本模版提供的环境如\autoref{table_ProvidedEnvironment}所示。
\begin{table}[h]
	\caption{\emph{scuthesis}~提供的环境}
	\label{table_ProvidedEnvironment}
	\begin{tabular*}{\textwidth}{l@{\extracolsep{\fill}}p{0.6\textwidth}}
		\toprule
		\textbf{环境名称} & \textbf{功能描述} \\
		\midrule
		\verb|CHSabstract| & 中文摘要环境,用于填写中文摘要。\\
		\verb|ENGabstract| & 英文摘要环境,用于填写英文摘要。\\
		\verb|reference| & 参考文献环境,使自动生成的参考文献符合格式规范。\\
		\bottomrule
	\end{tabular*}
\end{table}


\begin{equation}
	\label{eqn_ola}
	e^{j\omega\phi}=\cos(\omega\phi)+j\sin(\omega\phi)
\end{equation}
