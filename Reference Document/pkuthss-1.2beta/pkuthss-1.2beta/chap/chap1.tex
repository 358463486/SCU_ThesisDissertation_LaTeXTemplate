\chapter{使用介绍}
	\section{重要文件}

	本文档所在目录下各重要文件如下:
	\begin{itemize}\denseenum
		\item \verb|pkuthss.cls|:pkuthss~文档类的类文件。
		\item \verb|pkuthss.def|:在~\verb|pkuthss.cls|~中使用的定义文件。
		\item \verb|sample.tex|:主文件,编译该文件即可。
		\item \verb|sample.pdf|:即本文档,由编译~\verb|sample.tex|~得到。
		\item \verb|makefile|:Makefile,用于使编译工作自动化。
		\item \verb|chap/|:文件夹,包含各章节内容:
		\begin{itemize}\denseenum
			\item \verb|copyright.tex|:版权声明部分\footnote%
			{%
				因为本文档的许可证限制,我们必须附上许可证的文本;
				但用户可能选择其它类型的版权声明,
				故~\texttt{license/}\linebreak[1]~目录不是必需的。
				一个可能更常用的版权声明已经放在此文件中,但被注释掉了,
				用户可以考虑使用那个版本。
				如果使用那个版本,就不再需要~\texttt{license/}~目录了。
			}。
			\item \verb|originauth.tex|:
				原创性声明和使用授权说明部分~\supercite{F11}。
		\end{itemize}
		\item \verb|img/|:文件夹,包含论文中所有图片:
		\begin{itemize}\denseenum
			\item \verb|makefile|:图片部分的~Makefile。
			\item \verb|pkulogo.ps|:北大校徽。
			\item \verb|pkuword.ps|:“北京大学”字样。
		\end{itemize}
	\end{itemize}

	\section{系统要求}

	正确编译需要以下几部分:
	\begin{itemize}\denseenum
		\item 一个基本的~\LaTeX{}~发行版。
		\item CJK~或~xeCJK(供~Xe\LaTeX{}~使用)宏包。
		\item ctex~宏包\supercite{ctex-doc,ctexfaq}%
			(提供了~ctexbook~文档类)。
		\item 中文字体。
		\item 如果需要使用~Makefile~来实现自动编译,还需要~Make~工具。
	\end{itemize}

	最新的~\TeX{}Live~系统和~\CTeX~套装都已经包含%
	除中文字体之外所有要求的项目;中文字体需要用户自行获得。

	Linux~用户可以从软件源获得~GNU~的~make;
	其它类~UNIX~系统应该也会提供~make~工具,请参阅相应的文档以获得帮助。
	Windows~用户可以从以下地址下载~Windows~下的~GNU make~工具:

	\url{http://gnuwin32.sourceforge.net/packages/make.htm}(国际网)
	\vspace{-0.1em}\par
	\url{http://c.pku.edu.cn/software/c/mingw-c.7z}\footnote%
	{\ 感谢曹东刚老师在教学网站提供~GNU make~的下载。}(北大校园网)

	为了获得最好的支持,我们建议用户使用最新版的~\LaTeX{}~系统和各宏包。

	\section{编译方式}

	pkuthss~文档模板支持三种编译方式,即
	\begin{itemize}\denseenum
	  \item \LaTeX{} -- dvipdf~方式:
		即顺次执行~latex,bibtex,latex,latex,dvipdfmx。
	  \item pdf\LaTeX{}~方式:
		即顺次执行 pdflatex,bibtex,pdflatex,pdflatex。
	  \item Xe\LaTeX{}~方式:
		即顺次执行 xelatex,bibtex,xelatex,xelatex。
	\end{itemize}

	pkuthss~文档模板附带的~Makefile~中已经对这三种编译方式进行了完整的配置。
	用户只需要在~Makefile~中通过设定变量~\verb|JOBNAME|~的值%
	指定被编译的主文件名,
	并通过设定变量~\verb|LATEX|~的值指定采用哪种编译方式,
	即可通过在主文件所在目录调用~Make~工具来实现自动编译:
	如果是在类~UNIX~环境下,则用户应该调用的命令名为~\verb|make|;
	而如果是在~Windows~环境下,
	则用户应该调用的命令名为~\verb|mingw32-make|。

